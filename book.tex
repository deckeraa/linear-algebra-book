%\documentclass[smallroyalvopaper,twoside,12pt]{memoir}  % stocksize 9.25 by 6.175 in
\documentclass[a4paper,twoside,12pt]{memoir}  % stocksize A4
\usepackage{lipsum,amsmath,systeme}
%\usepackage{showframe}
\settrimmedsize{9in}{6in}{*} % final page size after trimming
\setulmarginsandblock{1in}{1in}{*} % upper and lower margins defining typeblock height
% according to KDP width of typeblock no more than 5.375in 
% about 65 characters per line is good for reading. For many 12pt fonts
% this gives a line length of about 2.5 alphabet lengths as about 5.2in.
% For 10pt fonts the line length is about 4.5in. Lets say the line length
% is to be 5in which calls for a 0.625in outer margin.
\setlrmarginsandblock{0.375in}{0.625in}{*}

\checkandfixthelayout

\begin{document}
\section{What is Linear Algebra?}
You've probably done algebra before in your life.
You've taken an equation like $x + 2 = 5$ and solved it to determine that
the only value of $x$ that makes it true is $x = 3$.

There are some equations that have no solutions, like $x + 1 = x + 2$.
There are also some equations that have more than one solution, like $x^2 = 4$.
In that equation, both $x = -2$ and $x = 2$ work.

Equations can have more than one variable.
For example, if I were to ask you ``Suppose I have two numbers that add up to six'', I would be asking you
to solve the equation $x + y = 6$. (There's many solutions to this: 0 and 6, 1 and 5, 3.5 and 2.5, etc.).

When there is more than one variable, we can draw pretty graphs.
Compare what these two graphs look like: (remember, the graph of an equation is just all the points that satisfy the equation)

% TODO put a cute graph of y = 2x + 1 with y = x^2 + 1

There's a difference between these two equations -- one is a straight line, the other is curvy.
(To be precise, $y = x^2 + 1$ is a parabola).

In linear algebra, we only deal with linear equations.
We don't deal with equations that make parabolas or orther curves\footnote{I should note that many time when a math book uses the word ``curve'' it includes straight lines, since a straight line is a just a curve that doesn't bend at all. When I use the word curve here I mean to exclude straight lines and to only mean things that are, you know, \ldots\ curvy.}.

How can you tell which equation will be a straight line? The exponent on the variables give it away. If the exponent is a one or a zero\footnote{Quick reminder about exponents of zero: $a^0$ will always be $1$, for any value of $a$. One reason why it is this way is to look at the pattern $3^1 = 3, 3^2 = 9, 3^3 = 27$. Every time we increase the exponent, we multiply be $3$. Going the other way, we \textit{divide} by $3$, so $3^0 = \frac{3}{3} = 1$. Another reason is that otherwise the exponent rules would break: $x^{a+b} = x^a*x^b$ means that $5^2 = 5^{2+0} = 5^2*5^0$. Therefore $5^0$ must be 1.} it will be straight. If any variable has an exponent besides one or zero it will be curvy.

Examples:
\begin{list}{}
\item $x + 2 = 4$ is a linear equation.
\item $x^2 + 2 = 4$ is not a linear equation.
\item $x + y + z = 3$ is a linear equation.
\item $x + y^{1.1} + z = 3$ is not a linear equation.
\end{list}

There's one more thing that can cause an equation to be non-linear: if it contains a non-linear function, then the equation will also be non-linear. Consider the sine function:

%TODO insert graph of a sine function with an arrow show that it's not straight

That's not straight at all! Therefore, an equation like $5 \sin(x) = 5$ is non-linear.

Now, it may seem like there's no fun in linear algebra -- you probably already know how to solve an equation like $x + 2 = 4$, so where's the challenge?
In linear algebra, so solve \textit{multiple equations at once}.

For example, if I told you ``the sum of two numbers is 5, and their difference is 1, what are the numbers?'' you would want to be able to determine that the numbers are $3$ and $2$.

I could write that same problem as
\begin{equation*}
  \systeme{
    x + y = 5,
    x - y = 1
    }
\end{equation*}
We would say that $x = 3, y = 2$ \textit{solves the system} since it works for both of the equations.
(The little thing to the left of the equation is called a ``curly brace'' and is used to indicated that both of the equations are grouped together into a system.)

In the remainder of this chapter, we'll be learning how to solve linear systems of any size.
Then we'll go on to see what shapes linear systems represent, and some really cool and mysterious tie-ins with the rest of mathematics.

\subsection{Exercises}
Which of the following are linear equations?
\begin{list}{}
\item 1) $3 - x = 2$
\item 2) $3 + 4 -5 + x = 0$?
\item 3) $5 = 5$
\item 4) $0 = 5$
\item 5) $x + y = 2$
\item 6) $x^2 + y = 2$
\item 7) $5x + \cos(x) = 10$
\end{list}

\section{History of Linear Algebra}
%TODO

\section{Adding Equations Together}
Before we jump into 

\end{document}
